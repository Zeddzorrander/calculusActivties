\documentclass[handout,nooutcomes,noauthor,12pt]{Ximera}
\input{preamble.tex}
\addtolength{\oddsidemargin}{-.5in}
\addtolength{\evensidemargin}{-.5in}
\addtolength{\textwidth}{1in}

\addtolength{\topmargin}{-.5in}
\addtolength{\textheight}{1in}
\outcome{understand closeness in the context of the definition of a limit}

\author{Kevin James}

\title{Definition of Limit Activity}

\begin{document}
	\begin{abstract}
		This activity is intended to help students gain a better understanding of "closeness" in the intuitive definition of a limit.
	\end{abstract}
	\maketitle
	\textbf{General Directions:}  Answer each question thoroughly.  Incorrect answers with work shown may receive partial credit, but unsubstantiated answers will receive NO CREDIT.  I do not want (decimal) approximations unless specifically asked for.  I want the exact numbers.  Justify all claims using calculus concepts (i.e., theorems, definitions, etc.).  I am looking for mathematical logic and reasoning.  Show all of your work!! Explain!  Explain!  Explain!
	
	Go to the In-Class Activities section of Blackboard and open the Definition of Limit activity.  
	
	
	\begin{enumerate}[label=\arabic*]
		\item Use the graph to estimate $ \lim\limits_{x \to 0}f(x) $.  Then update the values of $ L $ and $ c $ in the tool.    
		\begin{enumerate}[label=\alph*]
			\item Set Radius to $ 0.5 $.  Use the $ a $ and $ b $ sliders (or the points on the $ x $-axis) to find the largest interval, centered at $ x=0 $, such that $ f $ maps any $x$-value in this interval to within $ 0.5 $ units of $ L $.
			\vspace{\stretch{1}}
			\item Set Radius to $ 0.1 $.  Use the $ a $ and $ b $ sliders (or the points on the $ x $-axis) to find the largest interval, centered at $ x=0 $, such that $ f $ maps any $x$-value in this interval to within $ 0.1 $ units of $ L $.
			\vspace{\stretch{1}}
			\item Set Radius to $ 0.01 $.  Use the $ a $ and $ b $ sliders (or the points on the $ x $-axis) to find the largest interval, centered at $ x=0 $, such that $ f $ maps any $x$-value in this interval to within $ 0.01 $ units of $ L $.
			\vspace{\stretch{1}}
		\end{enumerate}
		\clearpage
		\item Now set $ L=1 $, but leave $ c $ set to $ 0 $.
		\begin{enumerate}[label=\alph*]
			\item Set Radius to $ 3 $.  Use the $ a $ and $ b $ sliders (or the points on the $ x $-axis) to find the largest interval, centered at $ x=0 $, such that $ f $ maps any $x$-value in this interval to within $ 3 $ units of $ L $.
			\vspace{1cm}
			\item Does this contradict that $ \lim\limits_{x \to 0}f(x)=4 $?  Explain your answer.
		\end{enumerate}

		\clearpage
		
		\item In this problem we will investigate $ \lim\limits_{x \to 1}f(x) $.  Notice, this limit doesn't exist since $ \lim\limits_{x \to 1^-}f(x) \neq \lim\limits_{x \to 1^+}f(x) $.  We will investigate three values for $ L $: $ 2, 3 $, and $ 2.5 $.
		\begin{enumerate}[label=\alph*]
			\item Let $ L=2, c=1 $, and Radius $ =1.5 $.  Use the $ a $ and $ b $ sliders (or the points on the $ x $-axis) to find the largest interval, centered at $ x=1 $, such that $ f $ maps any $x$-value in this interval to within $ 1.5 $ units of $ L $.
			\vspace{1.5cm}
			\begin{enumerate}
				\item Does this contradict that $ \lim\limits_{x \to 1}f(x) = DNE$?  Explain your answer.
				\vspace{\stretch{1}}
				\item Change Radius to $ 0.5 $.  Use the $ a $ and $ b $ sliders (or the points on the $ x $-axis) to find the largest interval, centered at $ x=1 $, such that $ f $ maps any $x$-value in this interval to within $ 0.5 $ units of $ L $.  What does this tell you?  Explain.
				\vspace{\stretch{1}}
			\end{enumerate}
		
			\clearpage
			
			\item Let $ L=3, c=1 $, and Radius $ =1.5 $.
			\begin{enumerate}
				\item Use the $ a $ and $ b $ sliders (or the points on the $ x $-axis) to find the largest interval, centered at $ x=1 $, such that $ f $ maps any $x$-value in this interval to within $ 1.5 $ units of $ L $.
				\vspace{\stretch{1}}
				\item Find a value for Radius such that there is no interval on the $x$-axis such that $x$-values get mapped to the interval created  by Radius.
				\vspace{\stretch{1}}
				\item Explain how this proves that $ \lim\limits_{x \to 1}f(x) \neq 3 $.
				\vspace{\stretch{1}}
			\end{enumerate}
		
			\clearpage
		
			\item Let $L=2.5$ and $c=1$.  Show that $ \lim\limits_{x \to 1}f(x) \neq 2.5 $.
			\vspace{\stretch{1}}
		\end{enumerate}
		\item Explain how the intervals in this activity connect to the concept of "closeness" in the intuitive definition of a limit.
		\vspace{\stretch{1}}
	\end{enumerate}
\end{document}
