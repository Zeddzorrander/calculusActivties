\documentclass[handout,nooutcomes,noauthor,12pt]{ximera}
\input{preamble.tex}
\addtolength{\oddsidemargin}{-.5in}
\addtolength{\evensidemargin}{-.5in}
\addtolength{\textwidth}{1in}

\addtolength{\topmargin}{-.5in}
\addtolength{\textheight}{1in}
\outcome{Understand and apply the Intermediate Value Theorem}

\author{Kevin James}

\title{The Intermediate Value Theorem}

\begin{document}
	\begin{abstract}
		We will investigate and apply the Intermediate Value Theorem.
	\end{abstract}
	\maketitle
	\textbf{General Directions:}  Answer each question thoroughly.  Incorrect answers with work shown may receive partial credit, but unsubstantiated answers will receive NO CREDIT.  I do not want (decimal) approximations unless specifically asked for.  I want the exact numbers.  Justify all claims using calculus concepts (i.e., theorems, definitions, etc.).  I am looking for mathematical logic and reasoning.  Show all of your work!! Explain!  Explain!  Explain!  \textbf{No graphing calculators are allowed for this activity.}
	
	\begin{theorem}[Intermediate Value Theorem]\label{theorem:IVT}\index{Intermediate Value Theorem}
		If $f$ is a continuous function for all $x$ in the closed interval
		$[a,b]$ and $d$ is between $f(a)$ and $f(b)$, then there is a number
		$c$ in $(a, b)$ such that $f(c) = d$.
	\end{theorem}
	
	\clearpage

	\begin{enumerate}[label=\arabic*]
		\item Given $ p(k) = k^3-3k+1 $ , we want to use the Intermediate Value Theorem to estimate the zero of $ p $ on $ [0, 1] $.
		\begin{enumerate}[label=\alph*]
			\item First, we must establish the conditions of the Intermediate Value Theorem (IVT) hold for $ p $ on $ [0, 1] $.
			\vspace{3cm}
			\item What value should we choose for $ d $ in the theorem if we want to find a root of $ p $?  Is this choice for $ d $ between $ p(0) $ and $ p(1) $?
			\vspace{3cm}
			\item Parts (a) and (b) allow us to use IVT to conclude that $ p $ \textbf{must} have a root on $ [0, 1] $.  To use IVT to find the root, we need to split the interval $ [0, 1] $ into two intervals: $ \left[0, \frac{1}{2}\right] $ and $ \left[\frac{1}{2}, 1\right] $.  Use IVT to determine which of these two intervals contains the root.  Make sure to establish the conditions of the theorem.  The conclusion alone is not sufficient.
			
			\clearpage
			
			\item Repeat the steps in part (c) until you have estimated the root to at least one decimal place acuracy.
		\end{enumerate}
	\end{enumerate}

\end{document}