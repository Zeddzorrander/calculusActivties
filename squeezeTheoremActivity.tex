\documentclass[handout,nooutcomes,noauthor]{Ximera}
\input{preamble.tex}
\addtolength{\oddsidemargin}{-.5in}
\addtolength{\evensidemargin}{-.5in}
\addtolength{\textwidth}{1in}

\addtolength{\topmargin}{-.5in}
\addtolength{\textheight}{1in}
\outcome{Understand the squeeze theorem}

\author{Kevin James}

\title{Squeeze Theorem Activity}

\begin{document}
	\begin{abstract}
		This activity is intended to help students understand the Squeeze Theorem."
	\end{abstract}
	\maketitle
	\textbf{Directions:}  This worksheet is to be used in conjunction with the Gradarius Assignment titled "Squeeze Theorem Activity".
	
	First, we start by presenting the Squeeze Theorem...
	
	\begin{theorem}[Squeeze Theorem]\index{Squeeze Theorem}
		Suppose that
		\[
		g(x) \le f(x) \le h(x)
		\]
		for all $x$ close to $a$ but not necessarily equal to $a$. If
		\[
		\lim_{x\to a} g(x) = L = \lim_{x\to a} h(x),
		\]
		then $\lim_{x\to a} f(x) = L$.
	\end{theorem}
	\vspace{\stretch{1}}
	\begin{enumerate}[label=\arabic*]
		\item State the conditions of the Squeeze Theorem for problem 1.
		\vspace{\stretch{1}}
		
		\clearpage
		
		\item State the conditions of the Squeeze Theorem for problem 2.
		\vspace{\stretch{1}}
		\item State the conditions of the Squeeze Theorem for problem 3.
		\vspace{\stretch{1}}
		\item Explanin why there are extra steps in problem 3 that we did not need in problem 2.
		\vspace{\stretch{1}}
	\end{enumerate}
\end{document}
	