\documentclass[handout,nooutcomes,noauthor,12pt]{ximera}
%\usepackage{todonotes}

\newcommand{\todo}{}

\usepackage{esint} % for \oiint
\ifxake%%https://math.meta.stackexchange.com/questions/9973/how-do-you-render-a-closed-surface-double-integral
\renewcommand{\oiint}{{\large\bigcirc}\kern-1.56em\iint}
\fi


\graphicspath{
  {./}
  {ximeraTutorial/}
  {basicPhilosophy/}
  {functionsOfSeveralVariables/}
  {normalVectors/}
  {lagrangeMultipliers/}
  {vectorFields/}
  {greensTheorem/}
  {shapeOfThingsToCome/}
  {dotProducts/}
  {partialDerivativesAndTheGradientVector/}
  {../productAndQuotientRules/exercises/}
  {../normalVectors/exercisesParametricPlots/}
  {../continuityOfFunctionsOfSeveralVariables/exercises/}
  {../partialDerivativesAndTheGradientVector/exercises/}
  {../directionalDerivativeAndChainRule/exercises/}
  {../commonCoordinates/exercisesCylindricalCoordinates/}
  {../commonCoordinates/exercisesSphericalCoordinates/}
  {../greensTheorem/exercisesCurlAndLineIntegrals/}
  {../greensTheorem/exercisesDivergenceAndLineIntegrals/}
  {../shapeOfThingsToCome/exercisesDivergenceTheorem/}
  {../greensTheorem/}
  {../shapeOfThingsToCome/}
  {../separableDifferentialEquations/exercises/}
  {vectorFields/}
}

\newcommand{\mooculus}{\textsf{\textbf{MOOC}\textnormal{\textsf{ULUS}}}}

\usepackage{tkz-euclide}\usepackage{tikz}
\usepackage{tikz-cd}
\usetikzlibrary{arrows}
\tikzset{>=stealth,commutative diagrams/.cd,
  arrow style=tikz,diagrams={>=stealth}} %% cool arrow head
\tikzset{shorten <>/.style={ shorten >=#1, shorten <=#1 } } %% allows shorter vectors

\usetikzlibrary{backgrounds} %% for boxes around graphs
\usetikzlibrary{shapes,positioning}  %% Clouds and stars
\usetikzlibrary{matrix} %% for matrix
\usepgfplotslibrary{polar} %% for polar plots
\usepgfplotslibrary{fillbetween} %% to shade area between curves in TikZ
\usetkzobj{all}
\usepackage[makeroom]{cancel} %% for strike outs
%\usepackage{mathtools} %% for pretty underbrace % Breaks Ximera
%\usepackage{multicol}
\usepackage{pgffor} %% required for integral for loops



%% http://tex.stackexchange.com/questions/66490/drawing-a-tikz-arc-specifying-the-center
%% Draws beach ball
\tikzset{pics/carc/.style args={#1:#2:#3}{code={\draw[pic actions] (#1:#3) arc(#1:#2:#3);}}}



\usepackage{array}
\setlength{\extrarowheight}{+.1cm}
\newdimen\digitwidth
\settowidth\digitwidth{9}
\def\divrule#1#2{
\noalign{\moveright#1\digitwidth
\vbox{\hrule width#2\digitwidth}}}





\newcommand{\RR}{\mathbb R}
\newcommand{\R}{\mathbb R}
\newcommand{\N}{\mathbb N}
\newcommand{\Z}{\mathbb Z}

\newcommand{\sagemath}{\textsf{SageMath}}


%\renewcommand{\d}{\,d\!}
\renewcommand{\d}{\mathop{}\!d}
\newcommand{\dd}[2][]{\frac{\d #1}{\d #2}}
\newcommand{\pp}[2][]{\frac{\partial #1}{\partial #2}}
\renewcommand{\l}{\ell}
\newcommand{\ddx}{\frac{d}{\d x}}

\newcommand{\zeroOverZero}{\ensuremath{\boldsymbol{\tfrac{0}{0}}}}
\newcommand{\inftyOverInfty}{\ensuremath{\boldsymbol{\tfrac{\infty}{\infty}}}}
\newcommand{\zeroOverInfty}{\ensuremath{\boldsymbol{\tfrac{0}{\infty}}}}
\newcommand{\zeroTimesInfty}{\ensuremath{\small\boldsymbol{0\cdot \infty}}}
\newcommand{\inftyMinusInfty}{\ensuremath{\small\boldsymbol{\infty - \infty}}}
\newcommand{\oneToInfty}{\ensuremath{\boldsymbol{1^\infty}}}
\newcommand{\zeroToZero}{\ensuremath{\boldsymbol{0^0}}}
\newcommand{\inftyToZero}{\ensuremath{\boldsymbol{\infty^0}}}



\newcommand{\numOverZero}{\ensuremath{\boldsymbol{\tfrac{\#}{0}}}}
\newcommand{\dfn}{\textbf}
%\newcommand{\unit}{\,\mathrm}
\newcommand{\unit}{\mathop{}\!\mathrm}
\newcommand{\eval}[1]{\bigg[ #1 \bigg]}
\newcommand{\seq}[1]{\left( #1 \right)}
\renewcommand{\epsilon}{\varepsilon}
\renewcommand{\phi}{\varphi}


\renewcommand{\iff}{\Leftrightarrow}

\DeclareMathOperator{\arccot}{arccot}
\DeclareMathOperator{\arcsec}{arcsec}
\DeclareMathOperator{\arccsc}{arccsc}
\DeclareMathOperator{\si}{Si}
\DeclareMathOperator{\scal}{scal}
\DeclareMathOperator{\sign}{sign}


%% \newcommand{\tightoverset}[2]{% for arrow vec
%%   \mathop{#2}\limits^{\vbox to -.5ex{\kern-0.75ex\hbox{$#1$}\vss}}}
\newcommand{\arrowvec}[1]{{\overset{\rightharpoonup}{#1}}}
%\renewcommand{\vec}[1]{\arrowvec{\mathbf{#1}}}
\renewcommand{\vec}[1]{{\overset{\boldsymbol{\rightharpoonup}}{\mathbf{#1}}}}

\newcommand{\point}[1]{\left(#1\right)} %this allows \vector{ to be changed to \vector{ with a quick find and replace
\newcommand{\pt}[1]{\mathbf{#1}} %this allows \vec{ to be changed to \vec{ with a quick find and replace
\newcommand{\Lim}[2]{\lim_{\point{#1} \to \point{#2}}} %Bart, I changed this to point since I want to use it.  It runs through both of the exercise and exerciseE files in limits section, which is why it was in each document to start with.

\DeclareMathOperator{\proj}{\mathbf{proj}}
\newcommand{\veci}{{\boldsymbol{\hat{\imath}}}}
\newcommand{\vecj}{{\boldsymbol{\hat{\jmath}}}}
\newcommand{\veck}{{\boldsymbol{\hat{k}}}}
\newcommand{\vecl}{\vec{\boldsymbol{\l}}}
\newcommand{\uvec}[1]{\mathbf{\hat{#1}}}
\newcommand{\utan}{\mathbf{\hat{t}}}
\newcommand{\unormal}{\mathbf{\hat{n}}}
\newcommand{\ubinormal}{\mathbf{\hat{b}}}

\newcommand{\dotp}{\bullet}
\newcommand{\cross}{\boldsymbol\times}
\newcommand{\grad}{\boldsymbol\nabla}
\newcommand{\divergence}{\grad\dotp}
\newcommand{\curl}{\grad\cross}
%\DeclareMathOperator{\divergence}{divergence}
%\DeclareMathOperator{\curl}[1]{\grad\cross #1}
\newcommand{\lto}{\mathop{\longrightarrow\,}\limits}

\renewcommand{\bar}{\overline}

\colorlet{textColor}{black}
\colorlet{background}{white}
\colorlet{penColor}{blue!50!black} % Color of a curve in a plot
\colorlet{penColor2}{red!50!black}% Color of a curve in a plot
\colorlet{penColor3}{red!50!blue} % Color of a curve in a plot
\colorlet{penColor4}{green!50!black} % Color of a curve in a plot
\colorlet{penColor5}{orange!80!black} % Color of a curve in a plot
\colorlet{penColor6}{yellow!70!black} % Color of a curve in a plot
\colorlet{fill1}{penColor!20} % Color of fill in a plot
\colorlet{fill2}{penColor2!20} % Color of fill in a plot
\colorlet{fillp}{fill1} % Color of positive area
\colorlet{filln}{penColor2!20} % Color of negative area
\colorlet{fill3}{penColor3!20} % Fill
\colorlet{fill4}{penColor4!20} % Fill
\colorlet{fill5}{penColor5!20} % Fill
\colorlet{gridColor}{gray!50} % Color of grid in a plot

\newcommand{\surfaceColor}{violet}
\newcommand{\surfaceColorTwo}{redyellow}
\newcommand{\sliceColor}{greenyellow}




\pgfmathdeclarefunction{gauss}{2}{% gives gaussian
  \pgfmathparse{1/(#2*sqrt(2*pi))*exp(-((x-#1)^2)/(2*#2^2))}%
}


%%%%%%%%%%%%%
%% Vectors
%%%%%%%%%%%%%

%% Simple horiz vectors
\renewcommand{\vector}[1]{\left\langle #1\right\rangle}


%% %% Complex Horiz Vectors with angle brackets
%% \makeatletter
%% \renewcommand{\vector}[2][ , ]{\left\langle%
%%   \def\nextitem{\def\nextitem{#1}}%
%%   \@for \el:=#2\do{\nextitem\el}\right\rangle%
%% }
%% \makeatother

%% %% Vertical Vectors
%% \def\vector#1{\begin{bmatrix}\vecListA#1,,\end{bmatrix}}
%% \def\vecListA#1,{\if,#1,\else #1\cr \expandafter \vecListA \fi}

%%%%%%%%%%%%%
%% End of vectors
%%%%%%%%%%%%%

%\newcommand{\fullwidth}{}
%\newcommand{\normalwidth}{}



%% makes a snazzy t-chart for evaluating functions
%\newenvironment{tchart}{\rowcolors{2}{}{background!90!textColor}\array}{\endarray}

%%This is to help with formatting on future title pages.
\newenvironment{sectionOutcomes}{}{}



%% Flowchart stuff
%\tikzstyle{startstop} = [rectangle, rounded corners, minimum width=3cm, minimum height=1cm,text centered, draw=black]
%\tikzstyle{question} = [rectangle, minimum width=3cm, minimum height=1cm, text centered, draw=black]
%\tikzstyle{decision} = [trapezium, trapezium left angle=70, trapezium right angle=110, minimum width=3cm, minimum height=1cm, text centered, draw=black]
%\tikzstyle{question} = [rectangle, rounded corners, minimum width=3cm, minimum height=1cm,text centered, draw=black]
%\tikzstyle{process} = [rectangle, minimum width=3cm, minimum height=1cm, text centered, draw=black]
%\tikzstyle{decision} = [trapezium, trapezium left angle=70, trapezium right angle=110, minimum width=3cm, minimum height=1cm, text centered, draw=black]

\addtolength{\oddsidemargin}{-.5in}
\addtolength{\evensidemargin}{-.5in}
\addtolength{\textwidth}{1in}

\addtolength{\topmargin}{-.5in}
\addtolength{\textheight}{1in}

\outcome{Compute average rate of change}
\outcome{Approximate instantaneous rate of change using average rates of change}
\outcome{Compute instataneous rates of change}

\author{Kevin James}

\title{An Application of Limits Activity}

\begin{document}
	\begin{abstract}
		
	\end{abstract}
	\maketitle
	\textbf{General Directions:}  Answer each question thoroughly.  Incorrect answers with work shown may receive partial credit, but unsubstantiated answers will receive NO CREDIT.  I do not want (decimal) approximations unless specifically asked for.  I want the exact numbers.  Justify all claims using calculus concepts (i.e., theorems, definitions, etc.).  I am looking for mathematical logic and reasoning.  Show all of your work!! Explain!  Explain!  Explain!
	
	Suppose the function $ s(t) = 16-(t-3)^2 $ represents the height $ s $ of a ball (in feet) at time $ t $ in seconds.  A graph of $ s $ is given below.
	
	\begin{image}
		\begin{tikzpicture}
		\begin{axis}[
		clip=false, domain=0:7, axis lines =middle, xlabel=$t$,
		ylabel=$s$, every axis y label/.style={at=(current
			axis.above origin),anchor=south}, every axis x
		label/.style={at=(current axis.right of
			origin),anchor=west}, ]
		\addplot [very thick, penColor,smooth] {16-(x-3)^2};
		%\addplot [very thick,penColor2] {80};
		% \addplot [very thick,penColor4] {20};
		\node at (axis cs:0,-1.2) {$0$};
		\end{axis}
		\end{tikzpicture}
	\end{image}
	
	 
	\begin{enumerate}[label=\arabic*.]
		\item First, let's see what we can estimate directly from the graph.
		\begin{enumerate}[label=(\alph*)]
			\item From what height is the ball intially thrown?
			\vspace{\stretch{1}}
			\item At what time does the ball hit the ground?
			\vspace{\stretch{1}}
			\item At what time does the ball reach its maximum height?
			\vspace{\stretch{1}}
			\item What is the maximum height of the ball?
		\end{enumerate}
	
		\clearpage
	\end{enumerate}	
	
	Reminder: $ s(t)=16-(t-3)^2 $
	
	\begin{enumerate}[resume,label=\arabic*.]
		\item Compute the average velocity of the ball on the interval $ [1, 2] $.
		\vspace{3cm}
		\item Compute the average velocity of the ball on the interval $ [2, 3] $
		\vspace{3cm}
		\item Compute the average velocity of the ball on the interval $ [2, 2+h] $ for $ h>0 $.  (Notice, when $ h<0 $, this becomes the interval $ [2+h, 2] $.)
		
		\clearpage
		
	\end{enumerate}
	
	Reminder: $ s(t)=16-(t-3)^2 $
	
	\begin{enumerate}[resume,label=\arabic*.]
		\item Use your answer to number 4 to compute the average velocity on each of the following intervals.
		\begin{enumerate}[label=(\alph*)]
			\item What is the average velocity on $ [2, 2.1] $?
			\vspace{3cm}
			\item What is the average velocity on $ [1.9, 2] $?
			\vspace{3cm}
			\item What is the average velocity on $ [2, 2.01] $?
			\vspace{3cm}
			\item What is the instantaneous velocity at $ t=2 $?
		\end{enumerate}
	\end{enumerate}

	\clearpage
	Recall from your reading:
	
	 \begin{definition}
		The \dfn{derivative} of $f$ at $a$, denoted $f'(a)$, is given by
		\[
		f'(a) = \lim_{x\to a} \frac{f(x) - f(a)}{x-a},
		\]
		provided that the limit exists. We say that $f$ is \dfn{differentiable}
		at $a$ if  this limit exists. Otherwise,  we say that  $f$ is \dfn{non-differentiable} at $a$.
	\end{definition}
	\vspace{1cm}
	
	This definition makes it clear that the derivative of $f$ at $x=a$ is the limit of the average rate of change (or the limit of the slope of the secant line between $(x, f(x))$ and $(a, f(a))$).  However, using this definition to compute the derivative of $f$ at $a$ is often inconvenient.  Therefore, we will use an alternative form of this definition.
	
	\begin{definition}
		The \dfn{derivative} of $f$ at $x=a$, denoted $f'(a)$, is given by 
		\[
			f'(a) = \lim_{h\to 0} \frac{f(a+h) - f(a)}{h}
		\]
		provided that the limit exists.
	\end{definition}
	\vspace{1cm}
	
	The image below shows that this definition is equivalent to the original one.  When computing derivatives, this second definition can be much easier to use.
	
	\begin{image}
		\begin{tikzpicture}
		\begin{axis}[
		domain=0:2, range=0:6,ymax=6,ymin=0,
		axis lines =left, xlabel=$x$, ylabel=$y$,
		every axis y label/.style={at=(current axis.above origin),anchor=south},
		every axis x label/.style={at=(current axis.right of origin),anchor=west},
		xtick={1,1.666}, ytick={1,3},
		xticklabels={$a$,$a+h$}, yticklabels={$f(a)$,$f(a+h)$},
		axis on top,
		]         
		% \addplot [penColor2!15!background, domain=(0:2)] {-3.348+4.348*x};
		%  \addplot [penColor2!32!background, domain=(0:2)] {-2.704+3.704*x};
		\addplot [very thick,penColor2, domain=(0:2)] {-1.994+2.994*x};         
		%  \addplot [penColor2!66!background, domain=(0:2)] {-1.326+2.326*x}; 
		% \addplot [penColor2!83!background, domain=(0:2)] {-0.666+1.666*x};
		\addplot [textColor,dashed] plot coordinates {(1,0) (1,0.2)};
		\addplot [textColor,dashed] plot coordinates {(0,1) (1,1)};
		\addplot [textColor,dashed] plot coordinates {(0,3) (1.666,3)};
		\addplot [textColor,dashed] plot coordinates {(1.666,0) (1.666,0.2)};
		\addplot [very thick,penColor, smooth,domain=(0:1.833)] {-1/(x-2)};
		\addplot[color=penColor,fill=penColor,only marks,mark=*] coordinates{(1.666,3)};  %% closed hole          
		\addplot[color=penColor,fill=penColor,only marks,mark=*] coordinates{(1,1)};  %% closed hole          
		 \addplot[decoration={brace,raise=.06cm},decorate,thin] plot coordinates
		{(1,0.1) (1.67,0.1)};
		% \addplot [very thick,penColor2, smooth,domain=(0:2)] {x};
		%  \addplot [very thick,penColor2,->]  plot coordinates {(1.666,1) (1.666,3)};
		% \addplot [very thick,penColor4,->]  plot coordinates {(1,1) (1.666,1)};
		\node at (axis cs:0.3,4.85) { \footnotesize secant line};
		\node at (axis cs:0.32,5.15) {\footnotesize slope of the};
		\node at (axis cs:1.2,5) {\footnotesize $=\frac{\Delta y}{\Delta x}=\frac{f(a+h)-f(a)}{h}$ };
		\node at (axis cs:0.8,1.35) { \footnotesize $(a,f(a))$};
		\node at (axis cs:1.3,3.5) { \footnotesize $(a+h,f(a+h))$};
		\node at (axis cs:1.34,0.5) {$h$};
		\node[color=penColor] at (axis cs:1.89,4.6){\large$f$};
		\end{axis}
		\end{tikzpicture}
	\end{image}
	\clearpage
	
	\begin{enumerate}[resume,label=\arabic*.]
		\item Compute $s'(5).$
	\end{enumerate}
	
\end{document}
