\documentclass[handout,nooutcomes,noauthor,12pt]{ximera}
\input{preamble.tex}
\addtolength{\oddsidemargin}{-.5in}
\addtolength{\evensidemargin}{-.5in}
\addtolength{\textwidth}{1in}

\addtolength{\topmargin}{-.5in}
\addtolength{\textheight}{1in}

\outcome{Compute average rate of change}
\outcome{Approximate instantaneous rate of change using average rates of change}
\outcome{Compute instataneous rates of change}

\author{Kevin James}

\title{An Application of Limits Activity}

\begin{document}
	\begin{abstract}
		
	\end{abstract}
	\maketitle
	\textbf{General Directions:}  Answer each question thoroughly.  Incorrect answers with work shown may receive partial credit, but unsubstantiated answers will receive NO CREDIT.  I do not want (decimal) approximations unless specifically asked for.  I want the exact numbers.  Justify all claims using calculus concepts (i.e., theorems, definitions, etc.).  I am looking for mathematical logic and reasoning.  Show all of your work!! Explain!  Explain!  Explain!
	
	Suppose the function $ s(t) = 16-(t-3)^2 $ represents the height $ s $ of a ball (in feet) at time $ t $ in seconds.  A graph of $ s $ is given below.
	
	\begin{image}
		\begin{tikzpicture}
		\begin{axis}[
		clip=false, domain=0:7, axis lines =middle, xlabel=$t$,
		ylabel=$s$, every axis y label/.style={at=(current
			axis.above origin),anchor=south}, every axis x
		label/.style={at=(current axis.right of
			origin),anchor=west}, ]
		\addplot [very thick, penColor,smooth] {16-(x-3)^2};
		%\addplot [very thick,penColor2] {80};
		% \addplot [very thick,penColor4] {20};
		\node at (axis cs:0,-1.2) {$0$};
		\end{axis}
		\end{tikzpicture}
	\end{image}
	
	 
	\begin{enumerate}[label=\arabic*.]
		\item First, let's see what we can estimate directly from the graph.
		\begin{enumerate}[label=(\alph*)]
			\item From what height is the ball intially thrown?
			\vspace{\stretch{1}}
			\item At what time does the ball hit the ground?
			\vspace{\stretch{1}}
			\item At what time does the ball reach its maximum height?
			\vspace{\stretch{1}}
			\item What is the maximum height of the ball?
		\end{enumerate}
	
		\clearpage
	\end{enumerate}	
	
	Reminder: $ s(t)=16-(t-3)^2 $
	
	\begin{enumerate}[resume,label=\arabic*.]
		\item Compute the average velocity of the ball on the interval $ [1, 2] $.
		\vspace{3cm}
		\item Compute the average velocity of the ball on the interval $ [2, 3] $
		\vspace{3cm}
		\item Compute the average velocity of the ball on the interval $ [2, 2+h] $ for $ h>0 $.  (Notice, when $ h<0 $, this becomes the interval $ [2+h, 2] $.)
		
		\clearpage
		
	\end{enumerate}
	
	Reminder: $ s(t)=16-(t-3)^2 $
	
	\begin{enumerate}[resume,label=\arabic*.]
		\item Use your answer to number 4 to compute the average velocity on each of the following intervals.
		\begin{enumerate}[label=(\alph*)]
			\item What is the average velocity on $ [2, 2.1] $?
			\vspace{3cm}
			\item What is the average velocity on $ [1.9, 2] $?
			\vspace{3cm}
			\item What is the average velocity on $ [2, 2.01] $?
			\vspace{3cm}
			\item What is the instantaneous velocity at $ t=2 $?
		\end{enumerate}
	\end{enumerate}

	\clearpage
	Recall from your reading:
	
	 \begin{definition}
		The \dfn{derivative} of $f$ at $a$, denoted $f'(a)$, is given by
		\[
		f'(a) = \lim_{x\to a} \frac{f(x) - f(a)}{x-a},
		\]
		provided that the limit exists. We say that $f$ is \dfn{differentiable}
		at $a$ if  this limit exists. Otherwise,  we say that  $f$ is \dfn{non-differentiable} at $a$.
	\end{definition}
	\vspace{1cm}
	
	This definition makes it clear that the derivative of $f$ at $x=a$ is the limit of the average rate of change (or the limit of the slope of the secant line between $(x, f(x))$ and $(a, f(a))$).  However, using this definition to compute the derivative of $f$ at $a$ is often inconvenient.  Therefore, we will use an alternative form of this definition.
	
	\begin{definition}
		The \dfn{derivative} of $f$ at $x=a$, denoted $f'(a)$, is given by 
		\[
			f'(a) = \lim_{h\to 0} \frac{f(a+h) - f(a)}{h}
		\]
		provided that the limit exists.
	\end{definition}
	\vspace{1cm}
	
	The image below shows that this definition is equivalent to the original one.  When computing derivatives, this second definition can be much easier to use.
	
	\begin{image}
		\begin{tikzpicture}
		\begin{axis}[
		domain=0:2, range=0:6,ymax=6,ymin=0,
		axis lines =left, xlabel=$x$, ylabel=$y$,
		every axis y label/.style={at=(current axis.above origin),anchor=south},
		every axis x label/.style={at=(current axis.right of origin),anchor=west},
		xtick={1,1.666}, ytick={1,3},
		xticklabels={$a$,$a+h$}, yticklabels={$f(a)$,$f(a+h)$},
		axis on top,
		]         
		% \addplot [penColor2!15!background, domain=(0:2)] {-3.348+4.348*x};
		%  \addplot [penColor2!32!background, domain=(0:2)] {-2.704+3.704*x};
		\addplot [very thick,penColor2, domain=(0:2)] {-1.994+2.994*x};         
		%  \addplot [penColor2!66!background, domain=(0:2)] {-1.326+2.326*x}; 
		% \addplot [penColor2!83!background, domain=(0:2)] {-0.666+1.666*x};
		\addplot [textColor,dashed] plot coordinates {(1,0) (1,0.2)};
		\addplot [textColor,dashed] plot coordinates {(0,1) (1,1)};
		\addplot [textColor,dashed] plot coordinates {(0,3) (1.666,3)};
		\addplot [textColor,dashed] plot coordinates {(1.666,0) (1.666,0.2)};
		\addplot [very thick,penColor, smooth,domain=(0:1.833)] {-1/(x-2)};
		\addplot[color=penColor,fill=penColor,only marks,mark=*] coordinates{(1.666,3)};  %% closed hole          
		\addplot[color=penColor,fill=penColor,only marks,mark=*] coordinates{(1,1)};  %% closed hole          
		 \addplot[decoration={brace,raise=.06cm},decorate,thin] plot coordinates
		{(1,0.1) (1.67,0.1)};
		% \addplot [very thick,penColor2, smooth,domain=(0:2)] {x};
		%  \addplot [very thick,penColor2,->]  plot coordinates {(1.666,1) (1.666,3)};
		% \addplot [very thick,penColor4,->]  plot coordinates {(1,1) (1.666,1)};
		\node at (axis cs:0.3,4.85) { \footnotesize secant line};
		\node at (axis cs:0.32,5.15) {\footnotesize slope of the};
		\node at (axis cs:1.2,5) {\footnotesize $=\frac{\Delta y}{\Delta x}=\frac{f(a+h)-f(a)}{h}$ };
		\node at (axis cs:0.8,1.35) { \footnotesize $(a,f(a))$};
		\node at (axis cs:1.3,3.5) { \footnotesize $(a+h,f(a+h))$};
		\node at (axis cs:1.34,0.5) {$h$};
		\node[color=penColor] at (axis cs:1.89,4.6){\large$f$};
		\end{axis}
		\end{tikzpicture}
	\end{image}
	\clearpage
	
	\begin{enumerate}[resume,label=\arabic*.]
		\item Compute $s'(5).$
	\end{enumerate}
	
\end{document}
