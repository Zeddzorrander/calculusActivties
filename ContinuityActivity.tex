\documentclass[handout,nooutcomes,noauthor,12pt]{ximera}
\input{preamble.tex}
\addtolength{\oddsidemargin}{-.5in}
\addtolength{\evensidemargin}{-.5in}
\addtolength{\textwidth}{1in}

\addtolength{\topmargin}{-.5in}
\addtolength{\textheight}{1in}
\outcome{Communicate fluently about continuity}

\author{Kevin James}

\title{Continuity Activity}

\begin{document}
	\begin{abstract}
		We will work on communicating clearly and precisely about the continuity of functions.
	\end{abstract}
	\maketitle
	\textbf{General Directions:}  Answer each question thoroughly.  Incorrect answers with work shown may receive partial credit, but unsubstantiated answers will receive NO CREDIT.  I do not want (decimal) approximations unless specifically asked for.  I want the exact numbers.  Justify all claims using calculus concepts (i.e., theorems, definitions, etc.).  I am looking for mathematical logic and reasoning.  Show all of your work!! Explain!  Explain!  Explain!
	
	\begin{problem}
		Given $ f(x) = \begin{cases}
		  \sqrt[5]{x} \sin\left(\frac{1}{x}\right) & x \ne 0 \\
		  0 & x=0
		\end{cases} $, answer the following.
		\begin{enumerate}
			\item Expalin what needs to be determined to show whether or not $f(x)$ is continuous at $x=0$.
			\vspace{\stretch{1}}
			\item Make an educated guess on whether or not $f(x)$ is continuous at $x=0$.  Explain your thinking.
			\vspace{\stretch{1}}
			
			\clearpage
			
			\item If you think $f(x)$ is continuous at $x=0$, then describe a plan to prove it.  If you think it is not continous, describe a plan to prove that.
			\vspace{\stretch{1}}
			\item Carry out your plan in detail and explain your whole process.
			\vspace{\stretch{1}}
		\end{enumerate}
	\end{problem}

	\clearpage
	
	\begin{problem}
		Given $ f(x) = \begin{cases}
		\sqrt[5]{x} \cdot g(x) & x \ne 0 \\
		0 & x=0
		\end{cases} $, answer the following.
		\begin{enumerate}
			\item Give examples of $g(x)$ such that $f$ is continuous at $x=0$.
			\vspace{\stretch{1}}
			\item Give examples of $g(x)$ such that $f$ is \textbf{not} continuous at $x=0$.
			\vspace{\stretch{1}}
		\end{enumerate}
		
	\end{problem}

\end{document}
