\documentclass[handout,nooutcomes,noauthor,12pt]{ximera}
\input{preamble.tex}
\addtolength{\oddsidemargin}{-.5in}
\addtolength{\evensidemargin}{-.5in}
\addtolength{\textwidth}{1in}

\addtolength{\topmargin}{-.5in}
\addtolength{\textheight}{1in}
\outcome{Work with limit laws}
\outcome{understanding dominance}

\author{Kevin James}

\title{Limit Laws and Continuity}

\begin{document}
	\begin{abstract}
		We will explore applications of limit laws.
	\end{abstract}
	\maketitle
	\textbf{General Directions:}  Answer each question thoroughly.  Incorrect answers with work shown may receive partial credit, but unsubstantiated answers will receive NO CREDIT.  I do not want (decimal) approximations unless specifically asked for.  I want the exact numbers.  Justify all claims using calculus concepts (i.e., theorems, definitions, etc.).  I am looking for mathematical logic and reasoning.  Show all of your work!! Explain!  Explain!  Explain!
	
	\begin{enumerate}[label=\arabic*.]
		\item If $f(x)$ is a polynomial function, then compute $\lim\limits_{x \to 0}x \cdot f(x)$.
		\vspace{\stretch{1}}
		\item If $f(x)$ is a continuous function, then compute $\lim\limits_{x \to 0}x \cdot f(x)$.
		\vspace{\stretch{1}}
		
		\clearpage
		
		\item Create a function $f(x)$ such that $\lim\limits_{x \to 0}x \cdot f(x) \ne 0$.
	\end{enumerate}

\end{document}
